\documentclass[11pt,a4paper,sans]{moderncv}   % 

\usepackage{scrextend}
\changefontsizes[11pt]{9mpt}

% moderncv 主题

\moderncvstyle{casual}                       % 选项参数是 ‘casual’, ‘classic’, ‘oldstyle’ 和 ’banking’
\moderncvcolor{blue}                          % 
\nopagenumbers{}                             % 消除注释以取消自动页码生成功能

% 字符编码
\usepackage[UTF8]{ctex}                   % 替换你正在使用的编码

% 调整页面
\usepackage[scale=0.89]{geometry}
\usepackage[firstyear=2016, lastyear=2018]{moderntimeline}
\makeatletter
\tikzset{
    tl@startyear/.append style={
        xshift=(0.5-\tl@startfraction)*\hintscolumnwidth,
        anchor=base
    }
}
\makeatother

%\setlength{\hintscolumnwidth}{3cm}           % 如果你希望改变日期栏的宽度
%\renewcommand*{\namefont}{\fontsize{30}{30}\mdseries\upshape}
%\renewcommand*{\sectionfont}{\fontsize{13}{13}\mdseries\upshape}



% 个人信息
\firstname{}
\familyname{屈轩}

\quote{求职意向: Golang 后台开发工程师 }
%\title{ 个人简历 }

%----------------------------------------------------------------------------------
%            内容
%----------------------------------------------------------------------------------
\begin{document}

\maketitle

\vspace*{-15mm}
%% 个人信息
\section {基本信息}	
\cvdoubleitem{性别:}{男}{年龄:}{28岁}	
\cvdoubleitem{学校:}{陕西科技大学}{专业:}{网络工程}
\cvdoubleitem{电话:}{17710791243}{邮箱:}{qu\_xuan@icloud.com}
%\cventry{学校:}{陕西科技大学}{}{}{}{}
%\cventry{专业:}{网络工程}{}{}{}{}
%\cventry{年龄:}{25岁}{}{}{}{}
%\cventry{电话:}{17710791243}{}{}{}{}
%\cventry{邮箱:}{qu\_xuan@icloud.com}{}{}{}{}
%\cventry{GitHub:}{\href{https://github.com/ioito}{https://github.com/ioito}}{}{}{}{}


\section{专业技能}
\cvitem{编程语言: }{ Golnag, C, X86汇编, PHP, Python, Shell. }
\cvitem{知识储备: }{ 正则表达式, 操作系统原理, Linux C环境编程, Makefile编写, 多线程多进程. }
\cvitem{工具使用: }{ Git, MySQL, Vim, Tmux, GCC, Ansible. }


\section{工作经历}
\tllabelcventry{2012.8}{2012.12}{2016.06-2018.3}{金山云网络科技有限公司}{}{}{}{
\begin{tightitemize}
        \item 负责公司自动化运维开发工作.
        \item 负责维护公司zabbix监控系统,支持业务监控开发工作.
\end{tightitemize}}
\tllabelcventry{2012.10}{2013.6}{2018.4-2020.11}{云联万维}{}{}{}{
\begin{tightitemize}
        \item 负责公司云平台公有云资源纳管功能. 
        \item 负责公司公有云认证服务开发.
\end{tightitemize}}


%\subsection{2016 - Now  金山云网络科技有限公司}
%\cvlistitem{负责公司自动化运维平台开发工作.}
%\cvlistitem{负责维护公司zabbix监控系统,支持业务监控开发工作.}


\section{ 项目经验 }

\subsection{ 微内核实现 }
\cvitem{项目描述: }{ 一个支持多进程多线程、中断处理、虚拟内存管理、虚拟文件系统、系统调用的x86微内核实现. }
\cvitem{项目职责: }{ 负责整个系统开发 }
\cvitem{实现原理: }{
    \itshape
    \begin{itemize}
    \item 由C及汇编实现,使用mutiboot2标准,由grub2加载至内存启动.
    \item 移植过vgabios汇编,实现32位显示模式切换.
    \item 虚拟文件系统支持挂载ext2、proc文件系统.
    \item 实现fork、exec、open、write等系统调用.
    \end{itemize}
    \upshape
}

\subsection{ 公有云认证统一登录 }
\cvitem{项目描述: }{ 支持公有云免认证一键登录. }
\cvitem{项目职责: }{ 负责认证,权限,跳转登录功能. }
\cvitem{实现原理: }{
    \itshape
    \begin{itemize}
    \item 借助公有云提供SAML认证方式.
    \item 配置用户相关权限.
    \item 使用SAML2.0协议登录跳转到公有云控制台.
    \end{itemize}
    \upshape
}

\subsection{ 公有云资源统一纳管 }
\cvitem{项目描述: }{ 纳管公有云资源. }
\cvitem{项目职责: }{ 负责公有云RDS, NAS, VM, WAF等资源纳管. }
\cvitem{实现原理: }{
    \itshape
    \begin{itemize}
    \item 借助公有云提供API, 获取公有云资源.
    \item 建立统一资源模型, 将公有云资源同步到本地.
    \item 支持公有云资源相应操作, 例如: 创建,删除,更新.
    \end{itemize}
    \upshape
}


\renewcommand{\baselinestretch}{1.0} 
 
\closesection{}                   % needed to renewcommands 
\renewcommand{\listitemsymbol}{-} % change the symbol for lists 
\clearpage

\end{document} 
%% 文件结尾 `template-zh.tex'.
